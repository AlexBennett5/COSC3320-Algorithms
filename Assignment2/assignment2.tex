\documentclass[11pt]{article}
\usepackage{amsmath}
\usepackage{minted}
\usepackage{xcolor}
\usepackage{parskip}

\title{Assignment 2 - COSC 3320}
\author{Alex Bennett (ID: 1901408)}
\date{}

\begin{document}
\maketitle

\section*{Theory Problem 1}

\subsection*{(a)}
To check if an element in $L_1$ is in $L_2$ we must iterate over each element of $L_1$ ($n$ elements), and then compare each element of $L_1$ with every element of $L_2$. Thus the lower bound is $O(n^2)$.

\subsection*{(b)}

\section*{Theory Problem 2}

\section*{Theory Problem 3}

To determine the average number of scalar multiplications for a sequence of $n$ matrices we will use the following informal algorithm:

$ S[i,i] = 0 \\ S[i,i+1] = p_i + p_{i+1} + p_{i+2} \\
S[i,j] = \operatorname{avg}(S[i,k] + S[k+1,j] + p_i + p_{k+1} + p_{k+1}), \text{ for } i \leq k \leq j-1 $

Where $S[i,j]$ is the matrix representing the average work for a parentheses configuration grouping every element from $i$ to $j$ and $p_i$ represents the $i$th dimension of the original matrix sequence. The average work is equal to the sum of scalar multiplications for possible k-values divided by the total number of k-values.

Since this algorithm will be iterated over 3 nested for-loops the time complexity is $O(n^3)$. Since $S$ is two dimensional our space complexity is $O(n^2)$

\section*{Programming Problem 1}

\subsection*{Procedure}

In this program we use three two-dimensional arrays $X, Y, \text{ and } Z$, each of size $n$, with $X \text{ and } Y$ initialized to 1 for each entry. We then add corresponding entries of $X \text{ and } Y$ together via matrix addition to generate an entry in $Z$. We use two different algorithms to perform the addition, the first traversing the arrays in row-major order and the second traversing them in column-major order, and compare their timings. The program is written in C++ and run on a MacOS supporting VMM.

\subsection*{Hypothesis}

Given that both algorithms have the same number of operations, one might expect that the timings for the two algorithms will be similar. However, as we've studied in class, if a multidimensional array is stored in memory using row-major order then row-major traversal is significantly faster than column-major traversal. Given that C++ must store these arrays in either row-major or column-major order, my hypothesis is that there will be significant difference in timing between the two algorithms.

\subsection*{C++ Implementation}

\begin{minted}
[
frame=lines,
framesep=2mm,
baselinestretch=1.2,
bgcolor=lightgray,
fontsize=\footnotesize,
linenos,
breaklines
]
{C}

#include <stdio.h>
#include <time.h>
#include <stdlib.h>

int main(int argc, char *argv[]) {

	char *arg = argv[1];
	int n = atoi(arg);

	int** X = new int*[n];
	int** Y = new int*[n];
	int** Z = new int*[n];

	//Initialize X, Y, Z
	for (int i = 0; i < n; i++) {
		X[i] = new int[n];
		Y[i] = new int[n];
		Z[i] = new int[n];
	}

	for (int i = 0; i < n; i++) {
		for (int j = 0; j < n; j++) {
			X[i][j] = 1;
			Y[i][j] = 1;
		}
	}
	
	//Version 1
	clock_t begin1 = clock();
	for (int i = 0; i < n; i++) {
		for (int j = 0; j < n; j++) {
			Z[i][j] = X[i][j] + Y[i][j];
		}
	}
	clock_t end1 = clock();

	//Version 2
	clock_t begin2 = clock();
	for (int j = 0; j < n; j++) {
		for (int i = 0; i < n; i++) {
			Z[i][j] = X[i][j] + Y[i][j];
		}
	}
	clock_t end2 = clock();

	int time1 = (double)(end1 - begin1)/(CLOCKS_PER_SEC/1000);
	int time2 = (double)(end2 - begin2)/(CLOCKS_PER_SEC/1000);

	printf("Version 1: %d milliseconds\n", time1);
	printf("Version 2: %d milliseconds\n", time2);

}
\end{minted}

\subsection*{Results}

\begin{center}
\begin{tabular}{| c | c c |}
\hline
Dimension of Array & Time of Vers. 1 (ms) & Time of Vers. 2 (ms) \\
\hline
128 & 0 & 0 \\
256 & 0 & 0 \\
512 & 0 & 3 \\
1024 & 5 & 22 \\
2048 & 20 &	108 \\
4096 & 81 & 545 \\
8192 & 325 & 2609 \\
16384 & 1421 & 15618 \\
32768 & 11147 & 109160 \\
65536 & 46973 & 824990 \\
\hline
\end{tabular}
\end{center}

\subsection*{Explanation}

My hypothesis was correct: there was a substantial difference in timing between Version 1 and Version 2. As explained in the hypothesis depending on how programming languages store arrays in memory, row-major and column-major traversals of an array can vary quite significantly in their timings. Since Version 1 (row-major) was much faster than Version 2 (column-major), we can conclude that C++ stores multidimensional arrays in row-major order. In fact this is true for most languages of the C family.

\section*{Programming Problem 2}

\subsection*{Procedure}

In this program we take an $n$ by $n$ matrix, initializing each entry to zero, then perform $m$ simple operations (adding a random number between 0-100 to a random entry in the matrix) for $m = 1677721600$ and $m = 13421772800$. The program is written in Java and run on a MacOS supporting VMM.

\subsection*{Hypothesis}

The algorithm is fairly simple, performing $m$ computations on the same matrix. However since we will be dealing with fairly large matrices, I expect that for certain values of $n$ the program will exceed physical memory and thus require the VMM to transfer data to disk storage. My hypothesis is that the timing of the program will jump up dramatically for particular $n$, and all larger $n$ will take significantly more time.

\subsection*{Java Implementation}

\begin{minted}
[
frame=lines,
framesep=2mm,
baselinestretch=1.2,
bgcolor=lightgray,
fontsize=\footnotesize,
linenos,
breaklines
]
{Java}

import java.util.Random;

public class Prog2 {

	public static void main (String[] args) {

		int n = Integer.parseInt(args[0]);

		int[][] M = new int[n][n];

		for (int k = 0; k < n; k++) {
			for (int q = 0; q < n; q++) {
				M[k][q] = 0;
			}
		}

		long start1 = System.currentTimeMillis();
		long m_1 = 1677721600;
		Random r = new Random();

		for (long k = 0; k < m_1; k++) {
			int x = r.nextInt(100) + 1;
			int j = r.nextInt(n);
			int i = r.nextInt(n);
			M[i][j] = M[i][j] + x;
		}

		long end1 = System.currentTimeMillis();

		long start2 = System.currentTimeMillis();
		long m_2 = 13421772800L;

		for (long k = 0; k < m_2; k++) {
			int x = r.nextInt(100) + 1;
			int j = r.nextInt(n);
			int i = r.nextInt(n);
			M[i][j] = M[i][j] + x;
		}

		long end2 = System.currentTimeMillis();

		System.out.println("Time elalpsed for n=" + n + " for m_1: " + (end1 - start1) + " milliseconds");
		System.out.println("Time elalpsed for n=" + n + " for m_2: " + (end2 - start2) + " milliseconds");


	}
}

\end{minted}

\subsection*{Results}

\begin{center}
\makebox[\textwidth]{
\begin{tabular}{| c | c c |}
\hline
n & Time for m=1677721600 (ms) & Time for m=13421772800 (ms) \\ 
\hline
16 & 55336 & 430628 \\
64 & 55951 & 424845 \\
256 & 59988 & 445868 \\
1024 & 121313 & 795774 \\
4096 & 232810 & 1858610 \\
16384 & 286520 & 2290697 \\
\hline
\end{tabular}}
\end{center}

\subsection*{Explanation}

My hypothesis was correct: for the first three values of $n$ the program took roughly the same amount of time to execute. But with $n=1024$ the timing increased dramatically. Larger values of $n$ took significantly longer as well. Interestingly the jump between $n=1024$ and $n=4096$ was much greater than the jump between $n=4096$ and $n=16384$. As I predicted in the hypothesis, the program ran out of physical memory and had to utilize the VMM to transfer data stored in RAM into disk storage. This swapping time accounts for the increase in execution time for larger values of $n$.

\section*{Programming Problem 3}

\subsection*{Procedure}

\subsection*{Hypothesis}

\subsection*{Java Implementation}

\begin{minted}
[
frame=lines,
framesep=2mm,
baselinestretch=1.2,
bgcolor=lightgray,
fontsize=\footnotesize,
linenos,
breaklines
]
{Java}

import java.util.Random;

class Node {
	int val;
	int[] matrix;
	Node left;
	Node right;
	int height;
}

class AVLtree {
	int M_0 = (int) Math.pow(2, 20);
	int M_1 = (int) (Math.pow(2, 19) + Math.pow(2, 18));
	int M_2 = (int) (Math.pow(2, 18) + Math.pow(2, 17));
	int inserts = 0;
	int deletes = 0;
	int nodes = 0;
	Node root;

	public void AVLtree() {
		this.root = null;
	}

	public void AVL_insert(int a) {
		this.root = insert(a, this.root);
	}

	public void AVL_remove(int a) {
		this.root = remove(a, this.root);
	}

	public int getNodeCount() {
		return this.nodes;
	}

	public int getInserts() {
		return this.inserts;
	}

	public int getDeletes() {
		return this.deletes;
	}

	public void resetInsertCount() {
		this.inserts = 0;
	}

	private Node insert(int a, Node b) {
		int mod = a % 3;

		if (b == null) {
			this.inserts++;
			b = new Node();
			this.nodes++;

			b.height = 0;
			b.left = null;
			b.right = null;

			if (mod == 0) {
				b.matrix = new int[M_0];
			} else if (mod == 1) {
				b.matrix = new int[M_1];
			} else {
				b.matrix = new int[M_2];
			}
		} else if (a < b.val) {
			b.left = insert(a, b.left);

			if (height(b.left) - height(b.right) == 2) {
				if (a < b.left.val) b = singRightRotate(b);
				else b = doubRightRotate(b);
			}
		} else if (a > b.val) {
			b.right = insert(a, b.right);
			if (height(b.right) - height(b.left) == 2) {
				if (a > b.right.val) b = singLeftRotate(b);
				else b = doubLeftRotate(b);
			}
		}

		b.height = Math.max(height(b.left), height(b.right)) + 1;
		return b;
	}

	private Node singRightRotate(Node b) {
	
		if (b == null || b.left == null) {
			return b;
		} else {
			Node temp = b.left;
			b.left = temp.right;
			temp.right = b;
			b.height = Math.max(height(b.left), height(b.right)) + 1;
			temp.height = Math.max(height(temp.left), b.height) + 1;
			return temp;
		}
	}

	private Node singLeftRotate(Node b) {

		if (b == null || b.right == null) {
			return b;
		} else {
			Node temp = b.right;
			b.right = temp.left;
			temp.left = b;
			b.height = Math.max(height(b.left), height(b.right)) + 1;
			temp.height = Math.max(height(b.right), b.height) + 1;
			return temp;
		}
	}

	private Node doubLeftRotate(Node b) {
		b.right = singRightRotate(b.right);
		return singLeftRotate(b);
	}

	private Node doubRightRotate(Node b) {
		b.left = singLeftRotate(b.left);
		return singRightRotate(b);
	}

	private Node remove(int a, Node b) {
		Node temp;

		if (b == null) return null;
		else if (a < b.val) b.left = remove(a, b.left);
		else if (a > b.val) b.right = remove(a, b.right);
		else if (b.left != null && b.right != null) {
			temp = findMin(b.right);
			b.val = temp.val;
			b.right = remove(b.val, b.right);
		} else {
			temp = b;
			if (b.left == null) b = b.right;
			else if (b.right == null) b = b.left;

			this.nodes--;
			this.deletes++;
		}
		if (b == null) return b;

		b.height = Math.max(height(b.left), height(b.right)) + 1;

		if ((height(b.left) - height(b.right)) == 2) {
			if ((height(b.left.left) - height(b.left.right)) == 1) return singLeftRotate(b);
			else return doubLeftRotate(b);
		} else if ((height(b.right) - height(b.left)) == 2) {
			if ((height(b.right.right) - height(b.right.left)) == 1) return singRightRotate(b);
			else return doubRightRotate(b);
		}
		return b;
	}

	private Node findMin(Node b) {
		if (b == null) return null;
		else if (b.left == null) return b;
		else return findMin(b.left);
	}

	private int height(Node b) {
		return (b == null ? -1 : b.height);
	}
}

public class Prog3 {

	public static void main(String[] args) {
	
		AVLtree tree = new AVLtree();
		Random r = new Random();

		double begin = System.currentTimeMillis();

		while(tree.getNodeCount() < 50) {
			tree.AVL_insert(r.nextInt(299));
		}

		double initialTime = System.currentTimeMillis() - begin;

		double insertTime, removeTime;
		insertTime = removeTime = 0;

		tree.resetInsertCount();

		for (int i = 0; i < 100000; i++) {

			while (tree.getNodeCount() < 50) {
				begin = System.currentTimeMillis();
				tree.AVL_insert(r.nextInt(299));
				insertTime += (System.currentTimeMillis() - begin);
			}

			while (tree.getNodeCount() >= 50) {
				begin = System.currentTimeMillis();
				tree.AVL_remove(r.nextInt(299));
				removeTime += (System.currentTimeMillis() - begin);
			}
		}

		System.out.println("Average time for initial insertion: " + (initialTime/50) + " ms");
		System.out.println("Average time for insertions: " + (insertTime/tree.getInserts()) + " ms");
		System.out.println("Average time for removals: " + (removeTime/tree.getDeletes()) + " ms");
	}
}

\end{minted}

\subsection*{Results}

\begin{center}
\begin{tabular}{c c}
\hline
Average time for initial insertion & 1.8 ms \\
Average time for subsequent insertions & 0.3037 ms \\
Average time for removals & 0.03291 ms \\
\hline
\end{tabular}
\end{center}

\subsection*{Explanation}


\section*{Programming Problem 4}

\subsection*{Procedure}

\subsection*{Hypothesis}

\subsection*{Java Implementation}

\begin{minted}
[
frame=lines,
framesep=2mm,
baselinestretch=1.2,
bgcolor=lightgray,
fontsize=\footnotesize,
linenos,
breaklines
]
{Java}
import com.sun.management.OperatingSystemMXBean;
import java.lang.management.ManagementFactory;

public class Prog4 {

	public static void main (String[] args) {
		
		OperatingSystemMXBean opsys = ManagementFactory.getPlatformMXBean(OperatingSystemMXBean.class);
		System.out.println("Phys Mem: " + opsys.getFreePhysicalMemorySize());
		System.out.println("Virtual Mem: " + opsys.getCommittedVirtualMemorySize());
		System.out.println("Free swap space: " + opsys.getFreeSwapSpaceSize());
	
		double[] C = {0.5, 0.6, 0.7, 0.8, 0.9, 0.95, 0.99, 1.0, 1.01, 1.1, 1.5, 2, 5, 10, 50};
		long freeBytes = opsys.getFreePhysicalMemorySize();

		for (int i = 0; i < 15; i++) {
			System.out.println("-----------------------\n" + "Cache Size: " + C[i] + "*M");
			long start = System.currentTimeMillis();

			int numBytes = Math.abs((int)(C[i] * (freeBytes)));
			int size = numBytes/4;
			int[] testArr = new int[size];

			System.out.println("Phys Mem: " + opsys.getFreePhysicalMemorySize());
			System.out.println("Virtual Mem: " + opsys.getCommittedVirtualMemorySize());
			System.out.println("Free swap space: " + opsys.getFreeSwapSpaceSize());
			
			for (int j = 0; j < size; j++) {
				testArr[j] = i + 1;
			}
			for (int j = 0; j < size; j++) {
				testArr[j] -= 2;
			}
			
			System.out.println("Time elapsed: " + ((double) System.currentTimeMillis() - start) + " milliseconds");

		}
	
	}
}

\end{minted}


\subsection*{Results}


\begin{center}
\makebox[\textwidth]{
\begin{tabular}{| c | c c c c |}
\hline
Cache Size & Free Physical Mem (bytes) & Free Virtual Mem (bytes) & Free Swap Space (bytes) & Time (ms) \\ 
\hline
Start & 292106240 & 10460585984 & 766246912 & N/A \\
0.5*M & 142024704 & 10476437504 & 766246912 & 138 \\
0.6*M & 105308160 & 10485899264 & 766246912 & 98 \\
0.7*M & 92807168 & 10486038528 & 766246912 & 78 \\
0.8*M & 78573568 & 10487099392 & 766246912 & 89 \\
0.9*M & 78573568 & 10488160256 & 766246912 & 102 \\
0.95*M & 78573568 & 10488160256 & 766246912 & 98 \\
0.99*M & 78573568 & 10488160256 & 766246912 & 100 \\
1.0*M & 78573568 & 10496548864 & 766246912 & 99 \\
1.01*M & 78573568 & 10496548864 & 766246912 & 100 \\
1.1*M & 78573568 & 10496548864 & 766246912 & 116 \\
1.5*M & 21618688 & 10496548864 & 766246912 & 187 \\
2.0*M & 22835200 & 10496548864 & 766246912 & 246 \\
5.0*M & 23138304 & 10498646016 &  766246912 & 1072 \\
10.0*M & 23121920 & 10499694592 & 766246912 & 824 \\
50.0*M & 21336064 & 10499796992 & 766246912 & 663 \\
\hline
\end{tabular}}
\end{center}

\subsection*{Explanation}

\section*{Programming Problem 5}

\subsection*{Description of algorithm}

Our algorithm for generating an optimal Huffman code works as follows. We first take characters and their frequencies as input from the user. We create tree nodes for each character and its frequency, and store them in a priority queue sorted by frequency. We then take the two least frequent nodes in the queue and construct a new node connecting the two, whose frequency is derived from their two frequencies combined. This new node is added back into the queue. We continue this process until only one node remains. This resultant tree supplies our Huffman encoding.

\subsection*{Java Implementation}

\begin{minted}
[
frame=lines,
framesep=2mm,
baselinestretch=1.2,
bgcolor=lightgray,
fontsize=\footnotesize,
linenos,
breaklines
]
{Java}
import java.util.Scanner;
import java.util.Comparator;
import java.util.ArrayList;
import java.util.PriorityQueue;

class Node {
	char ch;
	int freq;
	Node left;
	Node right;
}

class HuffComparator implements Comparator<Node> {
	public int compare(Node a, Node b) {
		return a.freq - b.freq;
	}
}

public class Prog5 {

	public static void main(String[] args) {

		Scanner reader = new Scanner(System.in);
		ArrayList<Character> chars = new ArrayList<Character>();
		ArrayList<Integer> frequency = new ArrayList<Integer>();
		boolean check = true;
		
		while(check) {
			System.out.println("Enter character (to exit enter '0')");
			char elem = reader.nextLine().charAt(0);
			
			if (elem == '0') {
				check = false;
			} else {
				System.out.println("What is the frequency for the character?");
				int freq = Integer.parseInt(reader.nextLine());
				chars.add(elem);
				frequency.add(freq);
			}
		}
		
		PriorityQueue<Node> queue = new PriorityQueue<Node>(chars.size(), new HuffComparator());

		for (int i = 0; i < chars.size(); i++) {
			Node newNode = new Node();
			
			newNode.ch = chars.get(i);
			newNode.freq = frequency.get(i);
			newNode.right = null;
			newNode.left = null;

			queue.add(newNode);
		}
		
		Node root = null;

		while (queue.size() > 1) {

			Node a = queue.peek();
			queue.poll();
			Node b = queue.peek();
			queue.poll();

			Node c = new Node();

			c.freq = a.freq + b.freq;
			c.ch = '-';
			c.left = a;
			c.right = b;
			root = c;

			queue.add(c);

		}

		printCode(root, "");

	}

	public static void printCode(Node root, String code) {
		if (root.left == null && root.right == null && Character.isLetter(root.ch)) {
			System.out.println(root.ch + ": " + code);
			return;
		}

		printCode(root.left, code + "0");
		printCode(root.right, code + "1");
	}

}
\end{minted}

\subsection*{Sample Huffman Code}

For character|frequency input:

a|8
v|1
z|1
b|5
c|6
p|4
e|10

Our algorithm generates the Huffman code:

c: 00
a: 01
e: 10
b: 110
v: 11100
z: 11101
p: 1111

\end{document}
